In the preliminary research, information was searched for vulnerabilities in MikroTik RouterOS, to get a better understanding of the severity of the vulnerabilities and on how these vulnerabilities can be exploited. More research still needs to be done on the reproducability of these vulnerabilities and how these vulnerabilities can be detected. This is not a full list of vulnerabilities on RouterOS, but it does include some recent vulnerabilities with a high score on the CVSS scale.

\subsection{Vulnerabilities in RouterOS}
CVE-2018-7445 is a vulnerability with the maximum score of 10 following the CVSSv2 scale. The vulnerability involves a bug in the Server Message Block (SMB) service of RouterOS, which could cause a stack overflow. The overflow happens when before authentication, causing an unauthenticated hacker to be able to execute malicious code. According to Core Security, the exploit takes place in a function parsing NetBIOS names, receiving two stack allocated buffers. These buffers will be copied over to the destination buffer without any size validation on the original buffer. If the original buffers are larger than the destination buffer, a stack overflow will happen. With an appropriate payload, a hacker can use this to execute malicious code or even gain root \cite{CVE-2018-7445:CORESEC:2018}. The vulnerability has been solved by MikroTik in RouterOS 6.41.3/6.42rc27. A workaround is disabling the SMB functionality on the router \cite{CVE-2018-7445:CORESEC:2018}.

CVE-2018-1156 is a vulnerability in the license upgrade system of MikroTik routers, which can be used to trigger a buffer overflow on the router and allow a hacker to remotely execute code. The vulnerability could be exploited by a remotely authenticated user with the following request \cite{CVE2018-1156:Tenable:2018}:
\begin{verbatim}
    GET /ssl_conn.php?usrname=%s&passwd=%s&softid
    =%s&level=%d&pay_type=%d&board=%d HTTP/1.0
\end{verbatim}

CVE-2018-14847 was originally published as a low priority bug where an attacker could only gain read only access to all files on the router. The bug is a directory traversal error in WinBox, an application from MikroTik to remotely access to router. A researcher from Tenable Research discovered that there was more to this vulnerability than expected. The vulnerability could also be used to write files on the file-system via WinBox or HTTP, without requiring any authentication \cite{CVE-2018-14847:TENABLE:2018}. This discovery increased the CVSSv2 score from a 5 to the maximum score of 10. Tenable estimates that of the vulnerable devices, 70\% will remain vulnerable \cite{CVE-2018-14847:TENABLE:2018}.

CVE-2019-3934 is a similar vulnerability to CVE-2018-14847. This is another directory traversal vulnerability in the WinBox software, which allows the hacker read and write access to all files on the router. The main difference with CVE-2018-14847 is that this vulnerability requires authentication before it can be invoked  \cite{CVE-2019-3943:TENABLE:2019}.