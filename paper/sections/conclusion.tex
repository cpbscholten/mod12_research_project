In this paper, attacks on low-cost routers were characterized by using a honeypot router to capture real hacker attacks. The paper discussed some common attacks targeting low-cost routers, such as DNS redirection attacks and code injection attacks. The honeypot router RouterOS from MikroTik was used, since RouterOS is a common vendor providing low-cost routers and in recent years multiple critical vulnerabilities were published for RouterOS. The honeypot was designed with the five principles for a secure honeypot in mind \cite{HONEYPOTSLIABILITY:SPRINGER:2015} and during the deployment of the honeypot all traffic and log files were captured.

With this approach, multiple different attacks were discovered. CVE-2018-14847 was the most common vulnerability targeted by attackers and was used to acquire the administrator credentials. These credentials were used to manually sign in and update the DNS server. Of all traffic on RouterOS specific port 8291, 71.1\% was related to this vulnerability. The attacks on CVE-2018-14847 were usually combined with untargeted DNS redirection attempts, which were 21.0\% of all traffic on port 8291 and 23.3\% of all traffic on the web interface at port 80. These DNS redirection attempts all originated from the Google Cloud Platform and followed the same characteristics. The other attacks include traffic from Internet of Things malware and a hacker updating the PPTP settings in the router. Furthermore, it was discovered that the router rebooted multiple times, usually after receiving one or multiple RST packets. 

The intents of the attackers could be characterized by either aiming to redirect traffic (DNS redirection / CVE-2018-14847), to intercept traffic (PPTP) or to deny traffic to the router (DoS / IoT Malware), with nearly all of the targeted attacks originating in cloud services.

\section{Discussion and Future work}
In this research, only attacks on low-cost routers from MikroTik were analyzed. In future research, honeypots simulating other brands of low-cost routers could be used to discover if there are differences in the characteristics of attacks between multiple vendors. Since this research was done by capturing attacks with a single honeypot in Brazil, more honeypots could be used per type of vendor and placed in different regions in order to detect regional differences in the discovered attacks.

Another improvement for future work could be to capture data for a longer period of time. This research was limited by the amount of research credits provided by Microsoft Azure, which limited the data capture to a short period of time. By collecting data for a longer period, it would become easier to discover patterns or trends in the observed attacks.