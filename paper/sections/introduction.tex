Network infrastructure devices such as routers are used to interconnect networks. Of these devices, low-cost routers have been popular in developing countries where these low-cost devices are used for expanding internet coverage in remote places. Low-cost routers are cheap routers which can be used as a home router or as a local area router with more advanced routing features such as the Border Gateway Protocol (BGP). These devices have been a popular target for hackers with these attacks becoming more popular as well. Recently, there has been a lot of regular news coverage about these devices actively being exploited. For example, the FBI discovered that hundreds of thousands of home routers were vulnerable against attacks from Russian hackers \cite{FBIRus:REUTERS:2018}.

There are many vendors providing low-cost routers. Some of these vendors include Huawei, TP-Link, NetGear and MikroTik. With over 1.6 million devices publicly visible \cite{MIKROTIK:SHODAN:2019}, MikroTik is a popular manufacturer of these low-cost routers. In recent years, multiple vulnerabilities for MikroTik routers classified as critical were discovered \cite{CVELIST}. These vulnerabilities have been a source for many attacks, one of which included two hundred thousand compromised MikroTik routers being used for mining cryptocurrency \cite{MikroTikCryptoHack:PCMAG:2018}. To protect systems and improve the security of the internet, it is important to characterize the attacks to such devices. By doing that, we can understand the intends of the hackers and set proper defenses. 

In this paper, attacks on low-cost routers will be characterized by using a honeypot router in a cloud environment to capture hacker attacks. A honeypot is a computer system designed with known vulnerabilities and can be used to detect attacks or to deflect attacks from a real target \cite{HONEYPOTDEF:NORTON}. The honeypot for this research will mimic RouterOS from MikroTik, which is an important subject to study due to the number of vulnerabilities published in recent years and the number of deployed devices in developing countries. MikroTik has different versions of its operational system RouterOS based on the device capabilities and MikroTik provides images of RouterOS for use in cloud environments. For this research a honeypot has been created with RouterOS in a cloud environment in order to characterize the different types of attacks against the honeypot. To be able to successfully detect hacker attacks against the honeypot, research will be done on the different attacks and how these attacks can be mapped.

\subsection{Research aims}
This research aims to answer the research questions mentioned below. These research questions are ordered in such a way that the that all previous questions need to be answered before the next question can be answered and build up to the goal of characterizing the attacks on low-cost routers.

\textbf{RQ1} Which types of attacks are low-cost routers vulnerable to and how can these attacks be mapped? \\
\textbf{RQ2} How can a honeypot be designed to discover attacks, while minimizing risks? \\
\textbf{RQ3} How can attacks on low-cost routers be characterized?